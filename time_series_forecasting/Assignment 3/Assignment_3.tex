\documentclass[]{article}
\usepackage{lmodern}
\usepackage{amssymb,amsmath}
\usepackage{ifxetex,ifluatex}
\usepackage{fixltx2e} % provides \textsubscript
\ifnum 0\ifxetex 1\fi\ifluatex 1\fi=0 % if pdftex
  \usepackage[T1]{fontenc}
  \usepackage[utf8]{inputenc}
\else % if luatex or xelatex
  \ifxetex
    \usepackage{mathspec}
  \else
    \usepackage{fontspec}
  \fi
  \defaultfontfeatures{Ligatures=TeX,Scale=MatchLowercase}
\fi
% use upquote if available, for straight quotes in verbatim environments
\IfFileExists{upquote.sty}{\usepackage{upquote}}{}
% use microtype if available
\IfFileExists{microtype.sty}{%
\usepackage{microtype}
\UseMicrotypeSet[protrusion]{basicmath} % disable protrusion for tt fonts
}{}
\usepackage[margin=1in]{geometry}
\usepackage{hyperref}
\hypersetup{unicode=true,
            pdftitle={Assignment 3},
            pdfborder={0 0 0},
            breaklinks=true}
\urlstyle{same}  % don't use monospace font for urls
\usepackage{color}
\usepackage{fancyvrb}
\newcommand{\VerbBar}{|}
\newcommand{\VERB}{\Verb[commandchars=\\\{\}]}
\DefineVerbatimEnvironment{Highlighting}{Verbatim}{commandchars=\\\{\}}
% Add ',fontsize=\small' for more characters per line
\usepackage{framed}
\definecolor{shadecolor}{RGB}{248,248,248}
\newenvironment{Shaded}{\begin{snugshade}}{\end{snugshade}}
\newcommand{\KeywordTok}[1]{\textcolor[rgb]{0.13,0.29,0.53}{\textbf{#1}}}
\newcommand{\DataTypeTok}[1]{\textcolor[rgb]{0.13,0.29,0.53}{#1}}
\newcommand{\DecValTok}[1]{\textcolor[rgb]{0.00,0.00,0.81}{#1}}
\newcommand{\BaseNTok}[1]{\textcolor[rgb]{0.00,0.00,0.81}{#1}}
\newcommand{\FloatTok}[1]{\textcolor[rgb]{0.00,0.00,0.81}{#1}}
\newcommand{\ConstantTok}[1]{\textcolor[rgb]{0.00,0.00,0.00}{#1}}
\newcommand{\CharTok}[1]{\textcolor[rgb]{0.31,0.60,0.02}{#1}}
\newcommand{\SpecialCharTok}[1]{\textcolor[rgb]{0.00,0.00,0.00}{#1}}
\newcommand{\StringTok}[1]{\textcolor[rgb]{0.31,0.60,0.02}{#1}}
\newcommand{\VerbatimStringTok}[1]{\textcolor[rgb]{0.31,0.60,0.02}{#1}}
\newcommand{\SpecialStringTok}[1]{\textcolor[rgb]{0.31,0.60,0.02}{#1}}
\newcommand{\ImportTok}[1]{#1}
\newcommand{\CommentTok}[1]{\textcolor[rgb]{0.56,0.35,0.01}{\textit{#1}}}
\newcommand{\DocumentationTok}[1]{\textcolor[rgb]{0.56,0.35,0.01}{\textbf{\textit{#1}}}}
\newcommand{\AnnotationTok}[1]{\textcolor[rgb]{0.56,0.35,0.01}{\textbf{\textit{#1}}}}
\newcommand{\CommentVarTok}[1]{\textcolor[rgb]{0.56,0.35,0.01}{\textbf{\textit{#1}}}}
\newcommand{\OtherTok}[1]{\textcolor[rgb]{0.56,0.35,0.01}{#1}}
\newcommand{\FunctionTok}[1]{\textcolor[rgb]{0.00,0.00,0.00}{#1}}
\newcommand{\VariableTok}[1]{\textcolor[rgb]{0.00,0.00,0.00}{#1}}
\newcommand{\ControlFlowTok}[1]{\textcolor[rgb]{0.13,0.29,0.53}{\textbf{#1}}}
\newcommand{\OperatorTok}[1]{\textcolor[rgb]{0.81,0.36,0.00}{\textbf{#1}}}
\newcommand{\BuiltInTok}[1]{#1}
\newcommand{\ExtensionTok}[1]{#1}
\newcommand{\PreprocessorTok}[1]{\textcolor[rgb]{0.56,0.35,0.01}{\textit{#1}}}
\newcommand{\AttributeTok}[1]{\textcolor[rgb]{0.77,0.63,0.00}{#1}}
\newcommand{\RegionMarkerTok}[1]{#1}
\newcommand{\InformationTok}[1]{\textcolor[rgb]{0.56,0.35,0.01}{\textbf{\textit{#1}}}}
\newcommand{\WarningTok}[1]{\textcolor[rgb]{0.56,0.35,0.01}{\textbf{\textit{#1}}}}
\newcommand{\AlertTok}[1]{\textcolor[rgb]{0.94,0.16,0.16}{#1}}
\newcommand{\ErrorTok}[1]{\textcolor[rgb]{0.64,0.00,0.00}{\textbf{#1}}}
\newcommand{\NormalTok}[1]{#1}
\usepackage{graphicx,grffile}
\makeatletter
\def\maxwidth{\ifdim\Gin@nat@width>\linewidth\linewidth\else\Gin@nat@width\fi}
\def\maxheight{\ifdim\Gin@nat@height>\textheight\textheight\else\Gin@nat@height\fi}
\makeatother
% Scale images if necessary, so that they will not overflow the page
% margins by default, and it is still possible to overwrite the defaults
% using explicit options in \includegraphics[width, height, ...]{}
\setkeys{Gin}{width=\maxwidth,height=\maxheight,keepaspectratio}
\IfFileExists{parskip.sty}{%
\usepackage{parskip}
}{% else
\setlength{\parindent}{0pt}
\setlength{\parskip}{6pt plus 2pt minus 1pt}
}
\setlength{\emergencystretch}{3em}  % prevent overfull lines
\providecommand{\tightlist}{%
  \setlength{\itemsep}{0pt}\setlength{\parskip}{0pt}}
\setcounter{secnumdepth}{0}
% Redefines (sub)paragraphs to behave more like sections
\ifx\paragraph\undefined\else
\let\oldparagraph\paragraph
\renewcommand{\paragraph}[1]{\oldparagraph{#1}\mbox{}}
\fi
\ifx\subparagraph\undefined\else
\let\oldsubparagraph\subparagraph
\renewcommand{\subparagraph}[1]{\oldsubparagraph{#1}\mbox{}}
\fi

%%% Use protect on footnotes to avoid problems with footnotes in titles
\let\rmarkdownfootnote\footnote%
\def\footnote{\protect\rmarkdownfootnote}

%%% Change title format to be more compact
\usepackage{titling}

% Create subtitle command for use in maketitle
\newcommand{\subtitle}[1]{
  \posttitle{
    \begin{center}\large#1\end{center}
    }
}

\setlength{\droptitle}{-2em}

  \title{Assignment 3}
    \pretitle{\vspace{\droptitle}\centering\huge}
  \posttitle{\par}
    \author{}
    \preauthor{}\postauthor{}
    \date{}
    \predate{}\postdate{}
  

\begin{document}
\maketitle

\begin{Shaded}
\begin{Highlighting}[]
\KeywordTok{library}\NormalTok{(tseries)}
\KeywordTok{library}\NormalTok{(zoo)}
\end{Highlighting}
\end{Shaded}

\begin{verbatim}
## 
## Attaching package: 'zoo'
\end{verbatim}

\begin{verbatim}
## The following objects are masked from 'package:base':
## 
##     as.Date, as.Date.numeric
\end{verbatim}

\begin{Shaded}
\begin{Highlighting}[]
\KeywordTok{library}\NormalTok{(ggplot2)}
\KeywordTok{library}\NormalTok{(forecast)}
\KeywordTok{library}\NormalTok{(expsmooth)}
\KeywordTok{library}\NormalTok{(fpp)}
\end{Highlighting}
\end{Shaded}

\begin{verbatim}
## Loading required package: fma
\end{verbatim}

\begin{verbatim}
## Loading required package: lmtest
\end{verbatim}

\subsection{Ch 8, Exercises 7, 9, 10,
11}\label{ch-8-exercises-7-9-10-11}

\subsubsection{\texorpdfstring{7. Consider `wmurders', the number of
women murdered each year (per 100,000 standard population) in the United
States.}{7. Consider wmurders, the number of women murdered each year (per 100,000 standard population) in the United States.}}\label{consider-wmurders-the-number-of-women-murdered-each-year-per-100000-standard-population-in-the-united-states.}

\paragraph{a. By studying appropriate graphs of the series in R, find an
appropriate ARIMA( p,d,q ) model for these
data.}\label{a.-by-studying-appropriate-graphs-of-the-series-in-r-find-an-appropriate-arima-pdq-model-for-these-data.}

\begin{Shaded}
\begin{Highlighting}[]
\KeywordTok{autoplot}\NormalTok{(wmurders)}
\end{Highlighting}
\end{Shaded}

\includegraphics{Assignment_3_files/figure-latex/unnamed-chunk-2-1.pdf}

\begin{Shaded}
\begin{Highlighting}[]
\NormalTok{wmurders }\OperatorTok\StringTok{ }\KeywordTok{ggtsdisplay}\NormalTok{()}
\end{Highlighting}
\end{Shaded}

\includegraphics{Assignment_3_files/figure-latex/unnamed-chunk-3-1.pdf}

\begin{Shaded}
\begin{Highlighting}[]
\NormalTok{wmurders }\OperatorTok\StringTok{ }\KeywordTok{diff}\NormalTok{() }\OperatorTok\StringTok{ }\KeywordTok{ggtsdisplay}\NormalTok{()}
\end{Highlighting}
\end{Shaded}

\includegraphics{Assignment_3_files/figure-latex/unnamed-chunk-4-1.pdf}

\begin{Shaded}
\begin{Highlighting}[]
\NormalTok{wmurders }\OperatorTok\StringTok{ }\KeywordTok{diff}\NormalTok{() }\OperatorTok\StringTok{ }\KeywordTok{diff}\NormalTok{() }\OperatorTok\StringTok{ }\KeywordTok{ggtsdisplay}\NormalTok{()}
\end{Highlighting}
\end{Shaded}

\includegraphics{Assignment_3_files/figure-latex/unnamed-chunk-5-1.pdf}

\paragraph{b. Should you include a constant in the model?
Explain.}\label{b.-should-you-include-a-constant-in-the-model-explain.}

No.

\paragraph{c. Write this model in terms of the backshift
operator.}\label{c.-write-this-model-in-terms-of-the-backshift-operator.}

\section{\texorpdfstring{(1 - B)\^{}2\emph{yt = (1 + theta1}B +
theta2\emph{B\^{}2)}et}{(1 - B)\^{}2yt = (1 + theta1B + theta2B\^{}2)et}}\label{b2yt-1-theta1b-theta2b2et}

\paragraph{d. Fit the model using R and examine the residuals. Is the
model
satisfactory?}\label{d.-fit-the-model-using-r-and-examine-the-residuals.-is-the-model-satisfactory}

\begin{Shaded}
\begin{Highlighting}[]
\NormalTok{wm_model_}\DecValTok{1}\NormalTok{ <-}\StringTok{ }\KeywordTok{Arima}\NormalTok{(wmurders,}\DataTypeTok{order=}\KeywordTok{c}\NormalTok{(}\DecValTok{0}\NormalTok{,}\DecValTok{2}\NormalTok{,}\DecValTok{2}\NormalTok{))}
\KeywordTok{summary}\NormalTok{(wm_model_}\DecValTok{1}\NormalTok{)}
\end{Highlighting}
\end{Shaded}

\begin{verbatim}
## Series: wmurders 
## ARIMA(0,2,2) 
## 
## Coefficients:
##           ma1     ma2
##       -1.0181  0.1470
## s.e.   0.1220  0.1156
## 
## sigma^2 estimated as 0.04702:  log likelihood=6.03
## AIC=-6.06   AICc=-5.57   BIC=-0.15
## 
## Training set error measures:
##                      ME      RMSE       MAE        MPE     MAPE      MASE
## Training set -0.0113461 0.2088162 0.1525773 -0.2403396 4.331729 0.9382785
##                     ACF1
## Training set -0.05094066
\end{verbatim}

\begin{Shaded}
\begin{Highlighting}[]
\KeywordTok{checkresiduals}\NormalTok{(wm_model_}\DecValTok{1}\NormalTok{)}
\end{Highlighting}
\end{Shaded}

\includegraphics{Assignment_3_files/figure-latex/unnamed-chunk-6-1.pdf}

\begin{verbatim}
## 
##  Ljung-Box test
## 
## data:  Residuals from ARIMA(0,2,2)
## Q* = 11.764, df = 8, p-value = 0.1621
## 
## Model df: 2.   Total lags used: 10
\end{verbatim}

\begin{Shaded}
\begin{Highlighting}[]
\NormalTok{wm_model_auto <-}\StringTok{ }\KeywordTok{auto.arima}\NormalTok{(wmurders)}
\KeywordTok{summary}\NormalTok{(wm_model_auto)}
\end{Highlighting}
\end{Shaded}

\begin{verbatim}
## Series: wmurders 
## ARIMA(1,2,1) 
## 
## Coefficients:
##           ar1      ma1
##       -0.2434  -0.8261
## s.e.   0.1553   0.1143
## 
## sigma^2 estimated as 0.04632:  log likelihood=6.44
## AIC=-6.88   AICc=-6.39   BIC=-0.97
## 
## Training set error measures:
##                       ME      RMSE       MAE        MPE     MAPE      MASE
## Training set -0.01065956 0.2072523 0.1528734 -0.2149476 4.335214 0.9400996
##                    ACF1
## Training set 0.02176343
\end{verbatim}

\begin{Shaded}
\begin{Highlighting}[]
\KeywordTok{checkresiduals}\NormalTok{(wm_model_auto)}
\end{Highlighting}
\end{Shaded}

\includegraphics{Assignment_3_files/figure-latex/unnamed-chunk-7-1.pdf}

\begin{verbatim}
## 
##  Ljung-Box test
## 
## data:  Residuals from ARIMA(1,2,1)
## Q* = 12.419, df = 8, p-value = 0.1335
## 
## Model df: 2.   Total lags used: 10
\end{verbatim}

\paragraph{e. Forecast three times
ahead.}\label{e.-forecast-three-times-ahead.}

\begin{Shaded}
\begin{Highlighting}[]
\NormalTok{wm_model_}\DecValTok{1} \OperatorTok\StringTok{ }\KeywordTok{forecast}\NormalTok{(}\DataTypeTok{h=}\DecValTok{3}\NormalTok{) }\OperatorTok\StringTok{ }\NormalTok{autoplot}
\end{Highlighting}
\end{Shaded}

\includegraphics{Assignment_3_files/figure-latex/unnamed-chunk-8-1.pdf}

\paragraph{f. Create a plot of the series with forecasts and prediction
intervals for the next three periods
shown.}\label{f.-create-a-plot-of-the-series-with-forecasts-and-prediction-intervals-for-the-next-three-periods-shown.}

\begin{Shaded}
\begin{Highlighting}[]
\NormalTok{wm_model_auto }\OperatorTok\StringTok{ }\KeywordTok{forecast}\NormalTok{(}\DataTypeTok{h=}\DecValTok{3}\NormalTok{) }\OperatorTok\StringTok{ }\NormalTok{autoplot}
\end{Highlighting}
\end{Shaded}

\includegraphics{Assignment_3_files/figure-latex/unnamed-chunk-9-1.pdf}

\paragraph{\texorpdfstring{g. Does `auto.arima()' give the same model
you have chosen? If not, which model do you think is
better?}{g. Does auto.arima() give the same model you have chosen? If not, which model do you think is better?}}\label{g.-does-auto.arima-give-the-same-model-you-have-chosen-if-not-which-model-do-you-think-is-better}

\subsubsection{\texorpdfstring{9. For the `usgdp'
series:}{9. For the usgdp series:}}\label{for-the-usgdp-series}

\paragraph{a. if necessary, find a suitable Box-Cox transformation for
the
data;}\label{a.-if-necessary-find-a-suitable-box-cox-transformation-for-the-data}

\begin{Shaded}
\begin{Highlighting}[]
\KeywordTok{autoplot}\NormalTok{(usgdp)}
\end{Highlighting}
\end{Shaded}

\includegraphics{Assignment_3_files/figure-latex/unnamed-chunk-10-1.pdf}

\begin{Shaded}
\begin{Highlighting}[]
\NormalTok{lambda_gdp <-}\StringTok{ }\KeywordTok{BoxCox.lambda}\NormalTok{(usgdp)}
\NormalTok{usgdp }\OperatorTok\StringTok{ }\KeywordTok{BoxCox}\NormalTok{(lambda_gdp) }\OperatorTok\StringTok{ }\KeywordTok{autoplot}\NormalTok{()}
\end{Highlighting}
\end{Shaded}

\includegraphics{Assignment_3_files/figure-latex/unnamed-chunk-10-2.pdf}

\paragraph{\texorpdfstring{b. fit a suitable ARIMA model to the
transformed data using
`auto.arima()';}{b. fit a suitable ARIMA model to the transformed data using auto.arima();}}\label{b.-fit-a-suitable-arima-model-to-the-transformed-data-using-auto.arima}

\begin{Shaded}
\begin{Highlighting}[]
\NormalTok{auto_fit <-}\StringTok{ }\KeywordTok{auto.arima}\NormalTok{(usgdp, }\DataTypeTok{lambda=}\NormalTok{lambda_gdp)}
\KeywordTok{summary}\NormalTok{(auto_fit)}
\end{Highlighting}
\end{Shaded}

\begin{verbatim}
## Series: usgdp 
## ARIMA(2,1,0) with drift 
## Box Cox transformation: lambda= 0.366352 
## 
## Coefficients:
##          ar1     ar2   drift
##       0.2795  0.1208  0.1829
## s.e.  0.0647  0.0648  0.0202
## 
## sigma^2 estimated as 0.03518:  log likelihood=61.56
## AIC=-115.11   AICc=-114.94   BIC=-101.26
## 
## Training set error measures:
##                    ME    RMSE      MAE         MPE      MAPE      MASE
## Training set 1.195275 39.2224 29.29521 -0.01363259 0.6863491 0.1655687
##                     ACF1
## Training set -0.03824844
\end{verbatim}

\begin{Shaded}
\begin{Highlighting}[]
\KeywordTok{checkresiduals}\NormalTok{(auto_fit)}
\end{Highlighting}
\end{Shaded}

\includegraphics{Assignment_3_files/figure-latex/unnamed-chunk-11-1.pdf}

\begin{verbatim}
## 
##  Ljung-Box test
## 
## data:  Residuals from ARIMA(2,1,0) with drift
## Q* = 6.5772, df = 5, p-value = 0.254
## 
## Model df: 3.   Total lags used: 8
\end{verbatim}

\begin{Shaded}
\begin{Highlighting}[]
\NormalTok{auto_fit2 <-}\StringTok{ }\KeywordTok{auto.arima}\NormalTok{(usgdp, }\DataTypeTok{lambda=}\NormalTok{lambda_gdp, }\DataTypeTok{stepwise =} \OtherTok{FALSE}\NormalTok{, }\DataTypeTok{approximation =} \OtherTok{FALSE}\NormalTok{)}
\KeywordTok{summary}\NormalTok{(auto_fit2)}
\end{Highlighting}
\end{Shaded}

\begin{verbatim}
## Series: usgdp 
## ARIMA(0,1,2) with drift 
## Box Cox transformation: lambda= 0.366352 
## 
## Coefficients:
##          ma1     ma2   drift
##       0.2727  0.2061  0.1829
## s.e.  0.0640  0.0609  0.0179
## 
## sigma^2 estimated as 0.03507:  log likelihood=61.94
## AIC=-115.89   AICc=-115.71   BIC=-102.03
## 
## Training set error measures:
##                    ME     RMSE      MAE         MPE      MAPE      MASE
## Training set 1.308821 39.24517 29.27606 -0.01627204 0.6801511 0.1654604
##                     ACF1
## Training set -0.02298287
\end{verbatim}

\begin{Shaded}
\begin{Highlighting}[]
\KeywordTok{checkresiduals}\NormalTok{(auto_fit2)}
\end{Highlighting}
\end{Shaded}

\includegraphics{Assignment_3_files/figure-latex/unnamed-chunk-12-1.pdf}

\begin{verbatim}
## 
##  Ljung-Box test
## 
## data:  Residuals from ARIMA(0,1,2) with drift
## Q* = 6.471, df = 5, p-value = 0.263
## 
## Model df: 3.   Total lags used: 8
\end{verbatim}

\paragraph{c. try some other plausible models by experimenting with the
orders
chosen;}\label{c.-try-some-other-plausible-models-by-experimenting-with-the-orders-chosen}

\begin{Shaded}
\begin{Highlighting}[]
\NormalTok{arima_fit1 <-}\StringTok{ }\KeywordTok{Arima}\NormalTok{(usgdp,}\DataTypeTok{order=}\KeywordTok{c}\NormalTok{(}\DecValTok{1}\NormalTok{,}\DecValTok{1}\NormalTok{,}\DecValTok{0}\NormalTok{), }\DataTypeTok{lambda=}\NormalTok{lambda_gdp)}
\KeywordTok{summary}\NormalTok{(arima_fit1)}
\end{Highlighting}
\end{Shaded}

\begin{verbatim}
## Series: usgdp 
## ARIMA(1,1,0) 
## Box Cox transformation: lambda= 0.366352 
## 
## Coefficients:
##          ar1
##       0.6326
## s.e.  0.0504
## 
## sigma^2 estimated as 0.04384:  log likelihood=34.39
## AIC=-64.78   AICc=-64.73   BIC=-57.85
## 
## Training set error measures:
##                    ME     RMSE      MAE       MPE      MAPE     MASE
## Training set 15.45449 45.49569 35.08393 0.3101283 0.7815664 0.198285
##                    ACF1
## Training set -0.3381619
\end{verbatim}

\begin{Shaded}
\begin{Highlighting}[]
\KeywordTok{checkresiduals}\NormalTok{(arima_fit1)}
\end{Highlighting}
\end{Shaded}

\includegraphics{Assignment_3_files/figure-latex/unnamed-chunk-13-1.pdf}

\begin{verbatim}
## 
##  Ljung-Box test
## 
## data:  Residuals from ARIMA(1,1,0)
## Q* = 32.515, df = 7, p-value = 3.259e-05
## 
## Model df: 1.   Total lags used: 8
\end{verbatim}

\begin{Shaded}
\begin{Highlighting}[]
\NormalTok{arima_fit2 <-}\StringTok{ }\KeywordTok{Arima}\NormalTok{(usgdp,}\DataTypeTok{order=}\KeywordTok{c}\NormalTok{(}\DecValTok{1}\NormalTok{,}\DecValTok{2}\NormalTok{,}\DecValTok{2}\NormalTok{), }\DataTypeTok{lambda=}\NormalTok{lambda_gdp)}
\KeywordTok{summary}\NormalTok{(arima_fit2)}
\end{Highlighting}
\end{Shaded}

\begin{verbatim}
## Series: usgdp 
## ARIMA(1,2,2) 
## Box Cox transformation: lambda= 0.366352 
## 
## Coefficients:
##          ar1     ma1     ma2
##       0.5233  -1.216  0.2172
## s.e.  0.1277   0.151  0.1390
## 
## sigma^2 estimated as 0.03561:  log likelihood=57.84
## AIC=-107.67   AICc=-107.5   BIC=-93.84
## 
## Training set error measures:
##                    ME     RMSE      MAE        MPE      MAPE      MASE
## Training set 3.662614 39.57938 29.43795 0.06625796 0.6817249 0.1663754
##                     ACF1
## Training set -0.07422567
\end{verbatim}

\begin{Shaded}
\begin{Highlighting}[]
\KeywordTok{checkresiduals}\NormalTok{(arima_fit2)}
\end{Highlighting}
\end{Shaded}

\includegraphics{Assignment_3_files/figure-latex/unnamed-chunk-14-1.pdf}

\begin{verbatim}
## 
##  Ljung-Box test
## 
## data:  Residuals from ARIMA(1,2,2)
## Q* = 8.8124, df = 5, p-value = 0.1168
## 
## Model df: 3.   Total lags used: 8
\end{verbatim}

\paragraph{d. choose what you think is the best model and check the
residual
diagnostics;}\label{d.-choose-what-you-think-is-the-best-model-and-check-the-residual-diagnostics}

\paragraph{e. produce forecasts of your fitted model. Do the forecasts
look
reasonable?}\label{e.-produce-forecasts-of-your-fitted-model.-do-the-forecasts-look-reasonable}

\begin{Shaded}
\begin{Highlighting}[]
\NormalTok{arima_fit1 }\OperatorTok\StringTok{ }\KeywordTok{forecast}\NormalTok{() }\OperatorTok\StringTok{ }\NormalTok{autoplot}
\end{Highlighting}
\end{Shaded}

\includegraphics{Assignment_3_files/figure-latex/unnamed-chunk-15-1.pdf}

\paragraph{\texorpdfstring{f. compare the results with what you would
obtain using `ets()' (with no
transformation).}{f. compare the results with what you would obtain using ets() (with no transformation).}}\label{f.-compare-the-results-with-what-you-would-obtain-using-ets-with-no-transformation.}

\begin{Shaded}
\begin{Highlighting}[]
\NormalTok{ets_gdp <-}\StringTok{ }\KeywordTok{ets}\NormalTok{(usgdp)}
\NormalTok{ets_gdp }\OperatorTok\StringTok{ }\KeywordTok{forecast}\NormalTok{() }\OperatorTok\StringTok{ }\KeywordTok{autoplot}\NormalTok{()}
\end{Highlighting}
\end{Shaded}

\includegraphics{Assignment_3_files/figure-latex/unnamed-chunk-16-1.pdf}

\subsubsection{\texorpdfstring{10. Consider `austourists', the quarterly
number of international tourists to Australia for the period
1999-2010.}{10. Consider austourists, the quarterly number of international tourists to Australia for the period 1999-2010.}}\label{consider-austourists-the-quarterly-number-of-international-tourists-to-australia-for-the-period-1999-2010.}

\paragraph{a. Describe the time plot.}\label{a.-describe-the-time-plot.}

\begin{Shaded}
\begin{Highlighting}[]
\NormalTok{austourists }\OperatorTok\StringTok{ }\KeywordTok{autoplot}\NormalTok{()}
\end{Highlighting}
\end{Shaded}

\includegraphics{Assignment_3_files/figure-latex/unnamed-chunk-17-1.pdf}

\begin{Shaded}
\begin{Highlighting}[]
\NormalTok{austourists }\OperatorTok\StringTok{ }\KeywordTok{ggtsdisplay}\NormalTok{()}
\end{Highlighting}
\end{Shaded}

\includegraphics{Assignment_3_files/figure-latex/unnamed-chunk-17-2.pdf}

\paragraph{b. What can you learn from the ACF
graph?}\label{b.-what-can-you-learn-from-the-acf-graph}

TYPE THE THINGS

\paragraph{c. What can you learn from the PACF
graph?}\label{c.-what-can-you-learn-from-the-pacf-graph}

TYPE MORE THINGS

\paragraph{d. Produce plots of the seasonally differenced data (1 -
B\^{}4)Yt. What model do these graphs
suggest?}\label{d.-produce-plots-of-the-seasonally-differenced-data-1---b4yt.-what-model-do-these-graphs-suggest}

\begin{Shaded}
\begin{Highlighting}[]
\NormalTok{adj_tour <-}\StringTok{ }\NormalTok{austourists }\OperatorTok\StringTok{ }\KeywordTok{diff}\NormalTok{(}\DataTypeTok{lag=}\DecValTok{4}\NormalTok{)}
\KeywordTok{ggtsdisplay}\NormalTok{(adj_tour)}
\end{Highlighting}
\end{Shaded}

\includegraphics{Assignment_3_files/figure-latex/unnamed-chunk-18-1.pdf}

\begin{Shaded}
\begin{Highlighting}[]
\NormalTok{arima_tour <-}\StringTok{ }\KeywordTok{arima}\NormalTok{(austourists,}\DataTypeTok{order =} \KeywordTok{c}\NormalTok{(}\DecValTok{1}\NormalTok{, }\DecValTok{1}\NormalTok{, }\DecValTok{0}\NormalTok{),}\DataTypeTok{seasonal =} \KeywordTok{c}\NormalTok{(}\DecValTok{1}\NormalTok{, }\DecValTok{1}\NormalTok{, }\DecValTok{0}\NormalTok{))}
\KeywordTok{summary}\NormalTok{(arima_tour)}
\end{Highlighting}
\end{Shaded}

\begin{verbatim}
## 
## Call:
## arima(x = austourists, order = c(1, 1, 0), seasonal = c(1, 1, 0))
## 
## Coefficients:
##           ar1     sar1
##       -0.3683  -0.5446
## s.e.   0.1407   0.1228
## 
## sigma^2 estimated as 6.232:  log likelihood = -101.12,  aic = 208.24
## 
## Training set error measures:
##                       ME     RMSE      MAE        MPE     MAPE      MASE
## Training set -0.02038705 2.362785 1.768027 -0.7125777 5.247316 0.2046774
##                    ACF1
## Training set -0.0360196
\end{verbatim}

\begin{Shaded}
\begin{Highlighting}[]
\KeywordTok{checkresiduals}\NormalTok{(arima_tour)}
\end{Highlighting}
\end{Shaded}

\includegraphics{Assignment_3_files/figure-latex/unnamed-chunk-19-1.pdf}

\begin{verbatim}
## 
##  Ljung-Box test
## 
## data:  Residuals from ARIMA(1,1,0)(1,1,0)[4]
## Q* = 3.1378, df = 6, p-value = 0.7914
## 
## Model df: 2.   Total lags used: 8
\end{verbatim}

\paragraph{\texorpdfstring{e. Does `auto.arima()' give the same model
that you chose? If not, which model do you think is
better?}{e. Does auto.arima() give the same model that you chose? If not, which model do you think is better?}}\label{e.-does-auto.arima-give-the-same-model-that-you-chose-if-not-which-model-do-you-think-is-better}

\begin{Shaded}
\begin{Highlighting}[]
\NormalTok{auto_tour <-}\StringTok{ }\KeywordTok{auto.arima}\NormalTok{(austourists)}
\KeywordTok{summary}\NormalTok{(auto_tour)}
\end{Highlighting}
\end{Shaded}

\begin{verbatim}
## Series: austourists 
## ARIMA(1,0,0)(1,1,0)[4] with drift 
## 
## Coefficients:
##          ar1     sar1   drift
##       0.4493  -0.5012  0.4665
## s.e.  0.1368   0.1293  0.1055
## 
## sigma^2 estimated as 5.606:  log likelihood=-99.47
## AIC=206.95   AICc=207.97   BIC=214.09
## 
## Training set error measures:
##                      ME     RMSE      MAE        MPE     MAPE      MASE
## Training set 0.03377709 2.188233 1.632832 -0.6731192 5.000182 0.5633341
##                    ACF1
## Training set -0.0525015
\end{verbatim}

\begin{Shaded}
\begin{Highlighting}[]
\KeywordTok{checkresiduals}\NormalTok{(auto_tour)}
\end{Highlighting}
\end{Shaded}

\includegraphics{Assignment_3_files/figure-latex/unnamed-chunk-20-1.pdf}

\begin{verbatim}
## 
##  Ljung-Box test
## 
## data:  Residuals from ARIMA(1,0,0)(1,1,0)[4] with drift
## Q* = 2.4752, df = 5, p-value = 0.7802
## 
## Model df: 3.   Total lags used: 8
\end{verbatim}

\paragraph{f. Write the model in terms of the backshift operator, then
without using the backshift
operator.}\label{f.-write-the-model-in-terms-of-the-backshift-operator-then-without-using-the-backshift-operator.}

\begin{Shaded}
\begin{Highlighting}[]
\NormalTok{auto_tour}\OperatorTok{$}\NormalTok{model}
\end{Highlighting}
\end{Shaded}

\begin{verbatim}
## $phi
## [1]  0.4493064  0.0000000  0.0000000 -0.5011549  0.2251721
## 
## $theta
## [1] 0 0 0 0
## 
## $Delta
## [1] 0 0 0 1
## 
## $Z
## [1] 1 0 0 0 0 0 0 0 1
## 
## $a
## [1]  0.4053419 -1.1729275  0.7905546 -0.3073085  0.0849466 22.3930370
## [7] 13.7317025 38.7731311 25.1151686
## 
## $P
##                [,1]          [,2]          [,3]          [,4]
##  [1,]  0.000000e+00 -2.259166e-17 -1.757459e-18 -1.256724e-17
##  [2,] -2.259166e-17  5.852303e-18  3.901042e-19 -6.437372e-19
##  [3,] -1.757459e-18  3.901025e-19  5.114796e-21  2.121607e-18
##  [4,] -1.256724e-17 -6.437364e-19  2.121607e-18 -5.066907e-19
##  [5,]  7.179264e-18 -5.493207e-20 -9.537682e-19  1.138297e-19
##  [6,]  3.188345e-17 -4.066149e-31  1.500197e-22  4.436701e-18
##  [7,]  1.514976e-17  2.310852e-21 -2.455724e-17  9.910307e-18
##  [8,]  2.591314e-17  1.903574e-17 -8.533154e-18  6.918036e-18
##  [9,] -4.265699e-17  2.259179e-17  1.757745e-18  1.253828e-17
##                [,5]          [,6]          [,7]          [,8]
##  [1,]  7.179264e-18  3.189134e-17  1.517364e-17  2.595461e-17
##  [2,] -5.493207e-20  5.266755e-34  2.396795e-34  1.903574e-17
##  [3,] -9.537680e-19 -2.610824e-36 -2.457172e-17 -8.534439e-18
##  [4,]  1.138297e-19  4.439797e-18  9.916912e-18  6.932980e-18
##  [5,] -1.629124e-35 -1.994829e-18  5.047142e-19 -3.118755e-18
##  [6,] -1.993147e-18 -8.859131e-18  2.241459e-18 -1.385054e-17
##  [7,]  5.055228e-19 -1.663737e-33  4.903018e-17 -3.679947e-20
##  [8,] -3.110610e-18 -1.154274e-33 -1.220107e-33 -3.798375e-17
##  [9,] -7.167094e-18  2.501277e-34 -3.974373e-34 -3.144308e-33
##                [,9]
##  [1,] -4.267602e-17
##  [2,]  2.259166e-17
##  [3,]  1.757459e-18
##  [4,]  1.256724e-17
##  [5,] -7.179264e-18
##  [6,] -3.188345e-17
##  [7,] -1.514976e-17
##  [8,] -2.591314e-17
##  [9,]  4.265699e-17
## 
## $T
##             [,1] [,2] [,3] [,4] [,5] [,6] [,7] [,8] [,9]
##  [1,]  0.4493064    1    0    0    0    0    0    0    0
##  [2,]  0.0000000    0    1    0    0    0    0    0    0
##  [3,]  0.0000000    0    0    1    0    0    0    0    0
##  [4,] -0.5011549    0    0    0    1    0    0    0    0
##  [5,]  0.2251721    0    0    0    0    0    0    0    0
##  [6,]  1.0000000    0    0    0    0    0    0    0    1
##  [7,]  0.0000000    0    0    0    0    1    0    0    0
##  [8,]  0.0000000    0    0    0    0    0    1    0    0
##  [9,]  0.0000000    0    0    0    0    0    0    1    0
## 
## $V
##       [,1] [,2] [,3] [,4] [,5] [,6] [,7] [,8] [,9]
##  [1,]    1    0    0    0    0    0    0    0    0
##  [2,]    0    0    0    0    0    0    0    0    0
##  [3,]    0    0    0    0    0    0    0    0    0
##  [4,]    0    0    0    0    0    0    0    0    0
##  [5,]    0    0    0    0    0    0    0    0    0
##  [6,]    0    0    0    0    0    0    0    0    0
##  [7,]    0    0    0    0    0    0    0    0    0
##  [8,]    0    0    0    0    0    0    0    0    0
##  [9,]    0    0    0    0    0    0    0    0    0
## 
## $h
## [1] 0
## 
## $Pn
##                [,1]          [,2]          [,3]          [,4]
##  [1,]  1.000000e+00 -6.567276e-18 -1.830241e-18 -5.171292e-18
##  [2,] -6.567276e-18  5.852303e-18  3.901042e-19 -6.437372e-19
##  [3,] -1.830241e-18  3.901025e-19  5.114796e-21  2.121607e-18
##  [4,] -5.171293e-18 -6.437364e-19  2.121607e-18 -5.066907e-19
##  [5,]  3.143023e-18 -5.493207e-20 -9.537682e-19  1.138297e-19
##  [6,] -3.988357e-18 -4.071897e-31  1.500197e-22  4.436701e-18
##  [7,]  9.832181e-19  2.310852e-21 -2.455724e-17  9.910307e-18
##  [8,] -6.264605e-18  1.903574e-17 -8.533154e-18  6.918036e-18
##  [9,] -3.568417e-17  2.259179e-17  1.757745e-18  1.253828e-17
##                [,5]          [,6]          [,7]          [,8]
##  [1,]  3.143023e-18 -3.980465e-18  1.007102e-18 -6.223137e-18
##  [2,] -5.493207e-20 -4.814825e-35  1.266932e-35  1.903574e-17
##  [3,] -9.537680e-19  0.000000e+00 -2.457172e-17 -8.534439e-18
##  [4,]  1.138297e-19  4.439797e-18  9.916912e-18  6.932980e-18
##  [5,]  0.000000e+00 -1.994829e-18  5.047142e-19 -3.118755e-18
##  [6,] -1.993147e-18 -8.859131e-18  2.241459e-18 -1.385054e-17
##  [7,]  5.055228e-19 -1.155558e-33  4.903018e-17 -3.679947e-20
##  [8,] -3.110610e-18  0.000000e+00 -7.642592e-34 -3.798375e-17
##  [9,] -7.167094e-18  0.000000e+00 -4.962180e-34 -3.368678e-33
##                [,9]
##  [1,] -3.570320e-17
##  [2,]  2.259166e-17
##  [3,]  1.757459e-18
##  [4,]  1.256724e-17
##  [5,] -7.179264e-18
##  [6,] -3.188345e-17
##  [7,] -1.514976e-17
##  [8,] -2.591314e-17
##  [9,]  4.265699e-17
\end{verbatim}

\paragraph{\texorpdfstring{(1 - ??1\emph{B)(1 - phis1}B)(1 - B\^{}4)(yt
- c\emph{t) = et \#\#\#\# c = drift}(1 - phi1)(1 - phis1)*m\^{}D =
1.7793}{(1 - ??1B)(1 - phis1B)(1 - B\^{}4)(yt - ct) = et \#\#\#\# c = drift(1 - phi1)(1 - phis1)*m\^{}D = 1.7793}}\label{b1---phis1b1---b4yt---ct-et-c-drift1---phi11---phis1md-1.7793}

\paragraph{\texorpdfstring{(1 - phi1\emph{B - phis1}B +
phi1\emph{phis1}B\^{}2)(1 - B\^{}4)(yt - c\emph{t) = \#\#\#\# (1 -
phi1}B - phis1\emph{B + phi1}phis1\emph{B\^{}2 - B\^{}4 + phi1}B\^{}5 +
phis1\emph{B\^{}5 - phi1}phis1\emph{B\^{}6)(yt - c}t) =
et}{(1 - phi1B - phis1B + phi1phis1B\^{}2)(1 - B\^{}4)(yt - ct) = \#\#\#\# (1 - phi1B - phis1B + phi1phis1B\^{}2 - B\^{}4 + phi1B\^{}5 + phis1B\^{}5 - phi1phis1B\^{}6)(yt - ct) = et}}\label{phi1b---phis1b-phi1phis1b21---b4yt---ct-1---phi1b---phis1b-phi1phis1b2---b4-phi1b5-phis1b5---phi1phis1b6yt---ct-et}

\paragraph{\texorpdfstring{yt = c + (phi1 + phis1)\emph{yt-1 -
phi1}phis1\emph{yt-2 + yt-4 - (phi1 + phis1)}yt-5 + phi1\emph{phis1}yt-6
+
et}{yt = c + (phi1 + phis1)yt-1 - phi1phis1yt-2 + yt-4 - (phi1 + phis1)yt-5 + phi1phis1yt-6 + et}}\label{yt-c-phi1-phis1yt-1---phi1phis1yt-2-yt-4---phi1-phis1yt-5-phi1phis1yt-6-et}

\paragraph{\texorpdfstring{yt = 1.7793 - 0.06\emph{yt-1 + 0.2496}yt-2 +
yt-4 + 0.06\emph{yt-5 - 0.2496}yt-6 +
et}{yt = 1.7793 - 0.06yt-1 + 0.2496yt-2 + yt-4 + 0.06yt-5 - 0.2496yt-6 + et}}\label{yt-1.7793---0.06yt-1-0.2496yt-2-yt-4-0.06yt-5---0.2496yt-6-et}

\subsubsection{\texorpdfstring{11. Consider `usmelec', the total net
generation of electricity (in billion kilowatt hours) by the U.S.
electric industry (monthly for the period January 1973 - June 2013). In
general there are two peaks per year: in mid-summer and
mid-winter.}{11. Consider usmelec, the total net generation of electricity (in billion kilowatt hours) by the U.S. electric industry (monthly for the period January 1973 - June 2013). In general there are two peaks per year: in mid-summer and mid-winter.}}\label{consider-usmelec-the-total-net-generation-of-electricity-in-billion-kilowatt-hours-by-the-u.s.-electric-industry-monthly-for-the-period-january-1973---june-2013.-in-general-there-are-two-peaks-per-year-in-mid-summer-and-mid-winter.}

\paragraph{a. Examine the 12-month moving average of this series to see
what kind of trend is
involved.}\label{a.-examine-the-12-month-moving-average-of-this-series-to-see-what-kind-of-trend-is-involved.}

\begin{Shaded}
\begin{Highlighting}[]
\NormalTok{meanelec  <-}\StringTok{ }\KeywordTok{rollmean}\NormalTok{(usmelec,}\DecValTok{12}\NormalTok{)}
\KeywordTok{autoplot}\NormalTok{(usmelec, }\DataTypeTok{series =} \StringTok{"Raw Data"}\NormalTok{) }\OperatorTok{+}
\StringTok{  }\KeywordTok{autolayer}\NormalTok{(meanelec, }\DataTypeTok{series =} \StringTok{"12 Month Avg."}\NormalTok{) }\OperatorTok{+}
\StringTok{  }\KeywordTok{ylab}\NormalTok{(}\KeywordTok{expression}\NormalTok{(}\KeywordTok{paste}\NormalTok{(}\StringTok{"Electricity (in billion kilowatt hours)"}\NormalTok{))) }\OperatorTok{+}\StringTok{ }
\StringTok{  }\KeywordTok{ggtitle}\NormalTok{(}\StringTok{"Monthly total net generation of electricity"}\NormalTok{) }\CommentTok{#+}
\end{Highlighting}
\end{Shaded}

\includegraphics{Assignment_3_files/figure-latex/unnamed-chunk-22-1.pdf}

\begin{Shaded}
\begin{Highlighting}[]
  \CommentTok{#scale_color_discrete(breaks = c("Raw Data", "12 Month Avg."))}
\end{Highlighting}
\end{Shaded}

\paragraph{b. Do the data need transforming? If so, find a suitable
transformation.}\label{b.-do-the-data-need-transforming-if-so-find-a-suitable-transformation.}

Yes, definitely (expand on this)

\begin{Shaded}
\begin{Highlighting}[]
\NormalTok{lambda_elec <-}\StringTok{ }\KeywordTok{BoxCox.lambda}\NormalTok{(usmelec)}
\end{Highlighting}
\end{Shaded}

\paragraph{c. Are the data stationary? If not, find an appropriate
differencing which yields stationary
data.}\label{c.-are-the-data-stationary-if-not-find-an-appropriate-differencing-which-yields-stationary-data.}

The data is not stationary, so we need to difference it. Based on this,
the data should probably be seasonally differenced, and potentially
first difference it.

\begin{Shaded}
\begin{Highlighting}[]
\NormalTok{usmelec }\OperatorTok\StringTok{ }\KeywordTok{ggtsdisplay}\NormalTok{()}
\end{Highlighting}
\end{Shaded}

\includegraphics{Assignment_3_files/figure-latex/unnamed-chunk-24-1.pdf}

\paragraph{d. Identify a couple of ARIMA models that might be useful in
describing the time series. Which of your models is the best according
to their AIC
values?}\label{d.-identify-a-couple-of-arima-models-that-might-be-useful-in-describing-the-time-series.-which-of-your-models-is-the-best-according-to-their-aic-values}

\begin{Shaded}
\begin{Highlighting}[]
\NormalTok{usmelec }\OperatorTok\StringTok{ }\KeywordTok{BoxCox}\NormalTok{(lambda_elec) }\OperatorTok\StringTok{ }\KeywordTok{diff}\NormalTok{(}\DataTypeTok{lag=}\DecValTok{12}\NormalTok{)  }\OperatorTok\StringTok{ }\KeywordTok{ggtsdisplay}\NormalTok{(}\DataTypeTok{main=}\StringTok{"Seasonal Difference"}\NormalTok{)}
\end{Highlighting}
\end{Shaded}

\includegraphics{Assignment_3_files/figure-latex/unnamed-chunk-25-1.pdf}

\begin{Shaded}
\begin{Highlighting}[]
\NormalTok{usmelec }\OperatorTok\StringTok{ }\KeywordTok{BoxCox}\NormalTok{(lambda_elec) }\OperatorTok\StringTok{ }\KeywordTok{diff}\NormalTok{(}\DataTypeTok{lag=}\DecValTok{12}\NormalTok{)  }\OperatorTok\StringTok{ }\KeywordTok{diff}\NormalTok{(}\DataTypeTok{lag=}\DecValTok{1}\NormalTok{) }\OperatorTok\StringTok{ }\KeywordTok{ggtsdisplay}\NormalTok{(}\DataTypeTok{main=}\StringTok{"w/ First Difference"}\NormalTok{)}
\end{Highlighting}
\end{Shaded}

\includegraphics{Assignment_3_files/figure-latex/unnamed-chunk-25-2.pdf}

\begin{Shaded}
\begin{Highlighting}[]
\CommentTok{# I think that ARIMA(0, 1, 2)(0, 1, 1)[12] with Box-Cox transformation model might describe the data well. I'll try ARIMA(0, 1, 3)(0, 1, 1)[12] with Box-Cox transformation model, too.}
\NormalTok{usmelec_arima1 <-}\StringTok{ }\KeywordTok{Arima}\NormalTok{(usmelec,}\DataTypeTok{order =} \KeywordTok{c}\NormalTok{(}\DecValTok{0}\NormalTok{, }\DecValTok{1}\NormalTok{, }\DecValTok{2}\NormalTok{),}\DataTypeTok{seasonal =} \KeywordTok{c}\NormalTok{(}\DecValTok{0}\NormalTok{, }\DecValTok{1}\NormalTok{, }\DecValTok{1}\NormalTok{),}\DataTypeTok{lambda =}\NormalTok{ lambda_elec)}
\KeywordTok{summary}\NormalTok{(usmelec_arima1)}
\end{Highlighting}
\end{Shaded}

\begin{verbatim}
## Series: usmelec 
## ARIMA(0,1,2)(0,1,1)[12] 
## Box Cox transformation: lambda= -0.4772402 
## 
## Coefficients:
##           ma1      ma2     sma1
##       -0.4275  -0.2570  -0.8583
## s.e.   0.0455   0.0458   0.0278
## 
## sigma^2 estimated as 3.686e-06:  log likelihood=2132.66
## AIC=-4257.31   AICc=-4257.22   BIC=-4240.96
## 
## Training set error measures:
##                      ME     RMSE      MAE       MPE     MAPE      MASE
## Training set -0.3499476 7.121762 5.194804 -0.150549 1.996946 0.5705842
##                     ACF1
## Training set -0.01270233
\end{verbatim}

\begin{Shaded}
\begin{Highlighting}[]
\KeywordTok{checkresiduals}\NormalTok{(usmelec_arima1)}
\end{Highlighting}
\end{Shaded}

\includegraphics{Assignment_3_files/figure-latex/unnamed-chunk-26-1.pdf}

\begin{verbatim}
## 
##  Ljung-Box test
## 
## data:  Residuals from ARIMA(0,1,2)(0,1,1)[12]
## Q* = 33.02, df = 21, p-value = 0.046
## 
## Model df: 3.   Total lags used: 24
\end{verbatim}

\begin{Shaded}
\begin{Highlighting}[]
\CommentTok{# ARIMA(0, 1, 2)(0, 1, 1)[12] with Box-Cox transformation model was the best.}
\end{Highlighting}
\end{Shaded}

\begin{Shaded}
\begin{Highlighting}[]
\NormalTok{usmelec_arima2 <-}\StringTok{ }\KeywordTok{Arima}\NormalTok{(usmelec,}\DataTypeTok{order =} \KeywordTok{c}\NormalTok{(}\DecValTok{0}\NormalTok{, }\DecValTok{1}\NormalTok{, }\DecValTok{3}\NormalTok{),}\DataTypeTok{seasonal =} \KeywordTok{c}\NormalTok{(}\DecValTok{0}\NormalTok{, }\DecValTok{1}\NormalTok{, }\DecValTok{1}\NormalTok{),}\DataTypeTok{lambda =}\NormalTok{ lambda_elec)}
\KeywordTok{summary}\NormalTok{(usmelec_arima2)}
\end{Highlighting}
\end{Shaded}

\begin{verbatim}
## Series: usmelec 
## ARIMA(0,1,3)(0,1,1)[12] 
## Box Cox transformation: lambda= -0.4772402 
## 
## Coefficients:
##           ma1      ma2      ma3     sma1
##       -0.4193  -0.2324  -0.0463  -0.8621
## s.e.   0.0475   0.0533   0.0447   0.0275
## 
## sigma^2 estimated as 3.683e-06:  log likelihood=2133.19
## AIC=-4256.38   AICc=-4256.24   BIC=-4235.94
## 
## Training set error measures:
##                     ME     RMSE      MAE        MPE     MAPE      MASE
## Training set -0.375076 7.119804 5.185129 -0.1582213 1.993314 0.5695215
##                     ACF1
## Training set -0.02512828
\end{verbatim}

\begin{Shaded}
\begin{Highlighting}[]
\KeywordTok{checkresiduals}\NormalTok{(usmelec_arima2)}
\end{Highlighting}
\end{Shaded}

\includegraphics{Assignment_3_files/figure-latex/unnamed-chunk-27-1.pdf}

\begin{verbatim}
## 
##  Ljung-Box test
## 
## data:  Residuals from ARIMA(0,1,3)(0,1,1)[12]
## Q* = 32.382, df = 20, p-value = 0.03939
## 
## Model df: 4.   Total lags used: 24
\end{verbatim}

\paragraph{e. Estimate the parameters of your best model and do
diagnostic testing on the residuals. Do the residuals resemble white
noise? If not, try to find another ARIMA model which fits
better.}\label{e.-estimate-the-parameters-of-your-best-model-and-do-diagnostic-testing-on-the-residuals.-do-the-residuals-resemble-white-noise-if-not-try-to-find-another-arima-model-which-fits-better.}

\begin{Shaded}
\begin{Highlighting}[]
\NormalTok{usmelec_arima1}
\end{Highlighting}
\end{Shaded}

\begin{verbatim}
## Series: usmelec 
## ARIMA(0,1,2)(0,1,1)[12] 
## Box Cox transformation: lambda= -0.4772402 
## 
## Coefficients:
##           ma1      ma2     sma1
##       -0.4275  -0.2570  -0.8583
## s.e.   0.0455   0.0458   0.0278
## 
## sigma^2 estimated as 3.686e-06:  log likelihood=2132.66
## AIC=-4257.31   AICc=-4257.22   BIC=-4240.96
\end{verbatim}

\begin{Shaded}
\begin{Highlighting}[]
\CommentTok{#theta1 = -0.4317, theta2 = -0.2552, phis1 = -0.8536}
\KeywordTok{checkresiduals}\NormalTok{(usmelec_arima1)}
\end{Highlighting}
\end{Shaded}

\includegraphics{Assignment_3_files/figure-latex/unnamed-chunk-28-1.pdf}

\begin{verbatim}
## 
##  Ljung-Box test
## 
## data:  Residuals from ARIMA(0,1,2)(0,1,1)[12]
## Q* = 33.02, df = 21, p-value = 0.046
## 
## Model df: 3.   Total lags used: 24
\end{verbatim}

\begin{Shaded}
\begin{Highlighting}[]
\CommentTok{# Ljung-Box test result shows that the residuals can be thought of as white noise. And they are normally distributed.}
\CommentTok{# I want to know what model was selected if I used auto.arima function. I'll try it.}
\NormalTok{usmelec_autoarima <-}\StringTok{ }\KeywordTok{auto.arima}\NormalTok{(usmelec,}\DataTypeTok{lambda =}\NormalTok{ lambda_elec)}
\NormalTok{usmelec_autoarima}
\end{Highlighting}
\end{Shaded}

\begin{verbatim}
## Series: usmelec 
## ARIMA(2,1,2)(2,1,2)[12] 
## Box Cox transformation: lambda= -0.4772402 
## 
## Coefficients:
##          ar1      ar2      ma1     ma2    sar1     sar2     sma1    sma2
##       1.1596  -0.3593  -1.5714  0.6360  0.5171  -0.1299  -1.3623  0.4580
## s.e.  0.2288   0.1082   0.2255  0.1861  0.2857   0.0597   0.2868  0.2496
## 
## sigma^2 estimated as 3.658e-06:  log likelihood=2136.57
## AIC=-4255.15   AICc=-4254.73   BIC=-4218.35
\end{verbatim}

\begin{Shaded}
\begin{Highlighting}[]
\CommentTok{# The result is ARIMA(2, 1, 4)(0, 0, 2)[12] with drift after Box-Cox transformation model. AIC is -4722. But I can't compare the AIC value with what I got above, because the number of differencing was different(Differencing changes the way the likelihood is computed).}
\KeywordTok{checkresiduals}\NormalTok{(usmelec_autoarima)}
\end{Highlighting}
\end{Shaded}

\includegraphics{Assignment_3_files/figure-latex/unnamed-chunk-28-2.pdf}

\begin{verbatim}
## 
##  Ljung-Box test
## 
## data:  Residuals from ARIMA(2,1,2)(2,1,2)[12]
## Q* = 27.17, df = 16, p-value = 0.03963
## 
## Model df: 8.   Total lags used: 24
\end{verbatim}

\begin{Shaded}
\begin{Highlighting}[]
\CommentTok{# And the residuals aren't like white noise. Therefore I'll choose ARIMA(0, 1, 2)(0, 1, 1)[12] with Box-Cox transformation model.}
\end{Highlighting}
\end{Shaded}

\paragraph{f. Forecast the next 15 years of electricity generation by
the U.S. electric industry. Get the latest figures from the EIA to check
the accuracy of your
forecasts.}\label{f.-forecast-the-next-15-years-of-electricity-generation-by-the-u.s.-electric-industry.-get-the-latest-figures-from-the-eia-to-check-the-accuracy-of-your-forecasts.}

\begin{Shaded}
\begin{Highlighting}[]
\NormalTok{fc_usmelec <-}\StringTok{ }\KeywordTok{forecast}\NormalTok{(usmelec_arima1,}\DataTypeTok{h =} \DecValTok{12}\OperatorTok{*}\DecValTok{15}\NormalTok{)}
\CommentTok{# Get the latest figures.}
\NormalTok{workdir <-}\StringTok{ "C:/Users/ca034330/Google Drive/Corey - School/!Fall 2018 B/BIA 6315 - Time Series and Forecasting/Assignment 3"}
\CommentTok{#workdir <- "D:/Google Drive/Corey - School/!Fall 2018 B/BIA 6315 - Time Series and Forecasting/Assignment 3"}
\KeywordTok{setwd}\NormalTok{(workdir)}
\NormalTok{usmelec.new <-}\StringTok{ }\KeywordTok{read.csv}\NormalTok{(}\StringTok{"MER_T07_02A.csv"}\NormalTok{, }\DataTypeTok{sep =} \StringTok{","}\NormalTok{)}
\CommentTok{# need to do data munging before using the data.}
\CommentTok{# make new columns Year, Month using YYYYMM column.}
\NormalTok{usmelec.new[, }\StringTok{"Year"}\NormalTok{] <-}\StringTok{ }\KeywordTok{as.numeric}\NormalTok{(}\KeywordTok{substr}\NormalTok{(usmelec.new[, }\StringTok{"YYYYMM"}\NormalTok{], }\DecValTok{1}\NormalTok{, }\DecValTok{4}\NormalTok{))}
\NormalTok{usmelec.new[, }\StringTok{"Month"}\NormalTok{] <-}\StringTok{ }\KeywordTok{as.numeric}\NormalTok{(}\KeywordTok{substr}\NormalTok{(usmelec.new[, }\StringTok{"YYYYMM"}\NormalTok{], }\DecValTok{5}\NormalTok{, }\DecValTok{6}\NormalTok{))}
\CommentTok{# make usmelec.new only have Year, Month and Value columns with net generation total data.}
\NormalTok{usmelec.new <-}\StringTok{ }\KeywordTok{subset}\NormalTok{(usmelec.new, Description }\OperatorTok{==}\StringTok{ "Electricity Net Generation Total, All Sectors"}\NormalTok{, }\DataTypeTok{select =} \KeywordTok{c}\NormalTok{(}\StringTok{"Year"}\NormalTok{, }\StringTok{"Month"}\NormalTok{, }\StringTok{"Value"}\NormalTok{))}
\CommentTok{# remove data if month is 13. They are old yearly data.}
\NormalTok{usmelec.new <-}\StringTok{ }\KeywordTok{subset}\NormalTok{(usmelec.new, Month }\OperatorTok{!=}\StringTok{ }\DecValTok{13}\NormalTok{)}
\CommentTok{# change the Value column data type to number. And divide the numbers by 1000 because the unit of the values in usmelec.new are Million KWh, not Billion KWh.}
\NormalTok{usmelec.new[, }\StringTok{"Value"}\NormalTok{] <-}\StringTok{ }\KeywordTok{as.numeric}\NormalTok{(}\KeywordTok{as.character}\NormalTok{(usmelec.new[, }\StringTok{"Value"}\NormalTok{]))}\OperatorTok{/}\DecValTok{1000}
\CommentTok{# as.numeric(usmelec.new[, "Value"]) yields wrong data. Need to recognize the letters as character first, and then change the type as number. }
\KeywordTok{head}\NormalTok{(usmelec.new)}
\end{Highlighting}
\end{Shaded}

\begin{verbatim}
##      Year Month    Value
## 7429 1973     1 160.2180
## 7430 1973     2 143.5387
## 7431 1973     3 148.1584
## 7432 1973     4 139.5894
## 7433 1973     5 147.3951
## 7434 1973     6 161.2437
\end{verbatim}

\begin{Shaded}
\begin{Highlighting}[]
\KeywordTok{tail}\NormalTok{(usmelec.new)}
\end{Highlighting}
\end{Shaded}

\begin{verbatim}
##      Year Month    Value
## 8016 2018     3 321.0150
## 8017 2018     4 301.7911
## 8018 2018     5 339.6709
## 8019 2018     6 372.3861
## 8020 2018     7 412.3828
## 8021 2018     8 410.4847
\end{verbatim}

\begin{Shaded}
\begin{Highlighting}[]
\CommentTok{# first observation was taken in January, 1973. Final observation was taken in October, 2017.}
\CommentTok{# make ts time series using usmelec.new Value column data.}
\NormalTok{usmelec.new.ts <-}\StringTok{ }\KeywordTok{ts}\NormalTok{(usmelec.new[, }\StringTok{"Value"}\NormalTok{], }\DataTypeTok{start =} \KeywordTok{c}\NormalTok{(}\DecValTok{1973}\NormalTok{, }\DecValTok{1}\NormalTok{), }\DataTypeTok{frequency =} \DecValTok{12}\NormalTok{)}
\KeywordTok{tail}\NormalTok{(usmelec.new.ts)}
\end{Highlighting}
\end{Shaded}

\begin{verbatim}
##           Mar      Apr      May      Jun      Jul      Aug
## 2018 321.0150 301.7911 339.6709 372.3861 412.3828 410.4847
\end{verbatim}

\begin{Shaded}
\begin{Highlighting}[]
\CommentTok{# final observation was taken in October, 2017 as expected. }
\CommentTok{# get accuracy for 4 years of forecast horizon.}
\NormalTok{usmelec.new.ts_next4years <-}\StringTok{ }\KeywordTok{subset}\NormalTok{(usmelec.new.ts, }\DataTypeTok{start =} \KeywordTok{length}\NormalTok{(usmelec) }\OperatorTok{+}\StringTok{ }\DecValTok{1}\NormalTok{,}\DataTypeTok{end =} \KeywordTok{length}\NormalTok{(usmelec) }\OperatorTok{+}\StringTok{ }\DecValTok{12}\OperatorTok{*}\DecValTok{4}\NormalTok{)}
\KeywordTok{accuracy}\NormalTok{(fc_usmelec, usmelec.new.ts_next4years)}
\end{Highlighting}
\end{Shaded}

\begin{verbatim}
##                      ME      RMSE       MAE       MPE     MAPE      MASE
## Training set -0.3499476  7.121762  5.194804 -0.150549 1.996946 0.5705842
## Test set     -6.8820578 12.610507 10.351050 -2.058210 2.986820 1.1369334
##                     ACF1 Theil's U
## Training set -0.01270233        NA
## Test set      0.59900383 0.3934206
\end{verbatim}

\begin{Shaded}
\begin{Highlighting}[]
\CommentTok{# plot the results}
\KeywordTok{autoplot}\NormalTok{(fc_usmelec, }\DataTypeTok{series =} \StringTok{"Forecasts"}\NormalTok{) }\OperatorTok{+}\StringTok{ }\KeywordTok{autolayer}\NormalTok{(usmelec.new.ts, }\DataTypeTok{series =} \StringTok{"Real data"}\NormalTok{) }\OperatorTok{+}\StringTok{ }\KeywordTok{scale_x_continuous}\NormalTok{(}\DataTypeTok{limits =} \KeywordTok{c}\NormalTok{(}\DecValTok{2010}\NormalTok{, }\DecValTok{2030}\NormalTok{)) }\OperatorTok{+}\StringTok{ }\KeywordTok{ggtitle}\NormalTok{(}\StringTok{"Forecast from ARIMA(0,1,2)(0,1,1)[12] with real data"}\NormalTok{)}
\end{Highlighting}
\end{Shaded}

\begin{verbatim}
## Scale for 'x' is already present. Adding another scale for 'x', which
## will replace the existing scale.
\end{verbatim}

\begin{verbatim}
## Warning: Removed 444 rows containing missing values (geom_path).

## Warning: Removed 444 rows containing missing values (geom_path).
\end{verbatim}

\includegraphics{Assignment_3_files/figure-latex/unnamed-chunk-29-1.pdf}

\begin{Shaded}
\begin{Highlighting}[]
\CommentTok{# Real data are really similar to the forecasts. Even when they were different, real data didn't get out of the prediction interval.}
\end{Highlighting}
\end{Shaded}

\paragraph{g. Eventually, the prediction intervals are so wide that the
forecasts are not particularly useful. How many years of forecasts do
you think are sufficiently accurate to be
usable?}\label{g.-eventually-the-prediction-intervals-are-so-wide-that-the-forecasts-are-not-particularly-useful.-how-many-years-of-forecasts-do-you-think-are-sufficiently-accurate-to-be-usable}

TEEEEXT HERE


\end{document}
